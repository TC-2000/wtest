
% Default to the notebook output style

    


% Inherit from the specified cell style.




    
\documentclass[11pt]{article}

    
    
    \usepackage[T1]{fontenc}
    % Nicer default font (+ math font) than Computer Modern for most use cases
    \usepackage{mathpazo}

    % Basic figure setup, for now with no caption control since it's done
    % automatically by Pandoc (which extracts ![](path) syntax from Markdown).
    \usepackage{graphicx}
    % We will generate all images so they have a width \maxwidth. This means
    % that they will get their normal width if they fit onto the page, but
    % are scaled down if they would overflow the margins.
    \makeatletter
    \def\maxwidth{\ifdim\Gin@nat@width>\linewidth\linewidth
    \else\Gin@nat@width\fi}
    \makeatother
    \let\Oldincludegraphics\includegraphics
    % Set max figure width to be 80% of text width, for now hardcoded.
    \renewcommand{\includegraphics}[1]{\Oldincludegraphics[width=.8\maxwidth]{#1}}
    % Ensure that by default, figures have no caption (until we provide a
    % proper Figure object with a Caption API and a way to capture that
    % in the conversion process - todo).
    \usepackage{caption}
    \DeclareCaptionLabelFormat{nolabel}{}
    \captionsetup{labelformat=nolabel}

    \usepackage{adjustbox} % Used to constrain images to a maximum size 
    \usepackage{xcolor} % Allow colors to be defined
    \usepackage{enumerate} % Needed for markdown enumerations to work
    \usepackage{geometry} % Used to adjust the document margins
    \usepackage{amsmath} % Equations
    \usepackage{amssymb} % Equations
    \usepackage{textcomp} % defines textquotesingle
    % Hack from http://tex.stackexchange.com/a/47451/13684:
    \AtBeginDocument{%
        \def\PYZsq{\textquotesingle}% Upright quotes in Pygmentized code
    }
    \usepackage{upquote} % Upright quotes for verbatim code
    \usepackage{eurosym} % defines \euro
    \usepackage[mathletters]{ucs} % Extended unicode (utf-8) support
    \usepackage[utf8x]{inputenc} % Allow utf-8 characters in the tex document
    \usepackage{fancyvrb} % verbatim replacement that allows latex
    \usepackage{grffile} % extends the file name processing of package graphics 
                         % to support a larger range 
    % The hyperref package gives us a pdf with properly built
    % internal navigation ('pdf bookmarks' for the table of contents,
    % internal cross-reference links, web links for URLs, etc.)
    \usepackage{hyperref}
    \usepackage{longtable} % longtable support required by pandoc >1.10
    \usepackage{booktabs}  % table support for pandoc > 1.12.2
    \usepackage[inline]{enumitem} % IRkernel/repr support (it uses the enumerate* environment)
    \usepackage[normalem]{ulem} % ulem is needed to support strikethroughs (\sout)
                                % normalem makes italics be italics, not underlines
    

    
    
    % Colors for the hyperref package
    \definecolor{urlcolor}{rgb}{0,.145,.698}
    \definecolor{linkcolor}{rgb}{.71,0.21,0.01}
    \definecolor{citecolor}{rgb}{.12,.54,.11}

    % ANSI colors
    \definecolor{ansi-black}{HTML}{3E424D}
    \definecolor{ansi-black-intense}{HTML}{282C36}
    \definecolor{ansi-red}{HTML}{E75C58}
    \definecolor{ansi-red-intense}{HTML}{B22B31}
    \definecolor{ansi-green}{HTML}{00A250}
    \definecolor{ansi-green-intense}{HTML}{007427}
    \definecolor{ansi-yellow}{HTML}{DDB62B}
    \definecolor{ansi-yellow-intense}{HTML}{B27D12}
    \definecolor{ansi-blue}{HTML}{208FFB}
    \definecolor{ansi-blue-intense}{HTML}{0065CA}
    \definecolor{ansi-magenta}{HTML}{D160C4}
    \definecolor{ansi-magenta-intense}{HTML}{A03196}
    \definecolor{ansi-cyan}{HTML}{60C6C8}
    \definecolor{ansi-cyan-intense}{HTML}{258F8F}
    \definecolor{ansi-white}{HTML}{C5C1B4}
    \definecolor{ansi-white-intense}{HTML}{A1A6B2}

    % commands and environments needed by pandoc snippets
    % extracted from the output of `pandoc -s`
    \providecommand{\tightlist}{%
      \setlength{\itemsep}{0pt}\setlength{\parskip}{0pt}}
    \DefineVerbatimEnvironment{Highlighting}{Verbatim}{commandchars=\\\{\}}
    % Add ',fontsize=\small' for more characters per line
    \newenvironment{Shaded}{}{}
    \newcommand{\KeywordTok}[1]{\textcolor[rgb]{0.00,0.44,0.13}{\textbf{{#1}}}}
    \newcommand{\DataTypeTok}[1]{\textcolor[rgb]{0.56,0.13,0.00}{{#1}}}
    \newcommand{\DecValTok}[1]{\textcolor[rgb]{0.25,0.63,0.44}{{#1}}}
    \newcommand{\BaseNTok}[1]{\textcolor[rgb]{0.25,0.63,0.44}{{#1}}}
    \newcommand{\FloatTok}[1]{\textcolor[rgb]{0.25,0.63,0.44}{{#1}}}
    \newcommand{\CharTok}[1]{\textcolor[rgb]{0.25,0.44,0.63}{{#1}}}
    \newcommand{\StringTok}[1]{\textcolor[rgb]{0.25,0.44,0.63}{{#1}}}
    \newcommand{\CommentTok}[1]{\textcolor[rgb]{0.38,0.63,0.69}{\textit{{#1}}}}
    \newcommand{\OtherTok}[1]{\textcolor[rgb]{0.00,0.44,0.13}{{#1}}}
    \newcommand{\AlertTok}[1]{\textcolor[rgb]{1.00,0.00,0.00}{\textbf{{#1}}}}
    \newcommand{\FunctionTok}[1]{\textcolor[rgb]{0.02,0.16,0.49}{{#1}}}
    \newcommand{\RegionMarkerTok}[1]{{#1}}
    \newcommand{\ErrorTok}[1]{\textcolor[rgb]{1.00,0.00,0.00}{\textbf{{#1}}}}
    \newcommand{\NormalTok}[1]{{#1}}
    
    % Additional commands for more recent versions of Pandoc
    \newcommand{\ConstantTok}[1]{\textcolor[rgb]{0.53,0.00,0.00}{{#1}}}
    \newcommand{\SpecialCharTok}[1]{\textcolor[rgb]{0.25,0.44,0.63}{{#1}}}
    \newcommand{\VerbatimStringTok}[1]{\textcolor[rgb]{0.25,0.44,0.63}{{#1}}}
    \newcommand{\SpecialStringTok}[1]{\textcolor[rgb]{0.73,0.40,0.53}{{#1}}}
    \newcommand{\ImportTok}[1]{{#1}}
    \newcommand{\DocumentationTok}[1]{\textcolor[rgb]{0.73,0.13,0.13}{\textit{{#1}}}}
    \newcommand{\AnnotationTok}[1]{\textcolor[rgb]{0.38,0.63,0.69}{\textbf{\textit{{#1}}}}}
    \newcommand{\CommentVarTok}[1]{\textcolor[rgb]{0.38,0.63,0.69}{\textbf{\textit{{#1}}}}}
    \newcommand{\VariableTok}[1]{\textcolor[rgb]{0.10,0.09,0.49}{{#1}}}
    \newcommand{\ControlFlowTok}[1]{\textcolor[rgb]{0.00,0.44,0.13}{\textbf{{#1}}}}
    \newcommand{\OperatorTok}[1]{\textcolor[rgb]{0.40,0.40,0.40}{{#1}}}
    \newcommand{\BuiltInTok}[1]{{#1}}
    \newcommand{\ExtensionTok}[1]{{#1}}
    \newcommand{\PreprocessorTok}[1]{\textcolor[rgb]{0.74,0.48,0.00}{{#1}}}
    \newcommand{\AttributeTok}[1]{\textcolor[rgb]{0.49,0.56,0.16}{{#1}}}
    \newcommand{\InformationTok}[1]{\textcolor[rgb]{0.38,0.63,0.69}{\textbf{\textit{{#1}}}}}
    \newcommand{\WarningTok}[1]{\textcolor[rgb]{0.38,0.63,0.69}{\textbf{\textit{{#1}}}}}
    
    
    % Define a nice break command that doesn't care if a line doesn't already
    % exist.
    \def\br{\hspace*{\fill} \\* }
    % Math Jax compatability definitions
    \def\gt{>}
    \def\lt{<}
    % Document parameters
    \title{The Notebook, but not that one starring Ryan Gosling }
    
    
    

    % Pygments definitions
    
\makeatletter
\def\PY@reset{\let\PY@it=\relax \let\PY@bf=\relax%
    \let\PY@ul=\relax \let\PY@tc=\relax%
    \let\PY@bc=\relax \let\PY@ff=\relax}
\def\PY@tok#1{\csname PY@tok@#1\endcsname}
\def\PY@toks#1+{\ifx\relax#1\empty\else%
    \PY@tok{#1}\expandafter\PY@toks\fi}
\def\PY@do#1{\PY@bc{\PY@tc{\PY@ul{%
    \PY@it{\PY@bf{\PY@ff{#1}}}}}}}
\def\PY#1#2{\PY@reset\PY@toks#1+\relax+\PY@do{#2}}

\expandafter\def\csname PY@tok@w\endcsname{\def\PY@tc##1{\textcolor[rgb]{0.73,0.73,0.73}{##1}}}
\expandafter\def\csname PY@tok@c\endcsname{\let\PY@it=\textit\def\PY@tc##1{\textcolor[rgb]{0.25,0.50,0.50}{##1}}}
\expandafter\def\csname PY@tok@cp\endcsname{\def\PY@tc##1{\textcolor[rgb]{0.74,0.48,0.00}{##1}}}
\expandafter\def\csname PY@tok@k\endcsname{\let\PY@bf=\textbf\def\PY@tc##1{\textcolor[rgb]{0.00,0.50,0.00}{##1}}}
\expandafter\def\csname PY@tok@kp\endcsname{\def\PY@tc##1{\textcolor[rgb]{0.00,0.50,0.00}{##1}}}
\expandafter\def\csname PY@tok@kt\endcsname{\def\PY@tc##1{\textcolor[rgb]{0.69,0.00,0.25}{##1}}}
\expandafter\def\csname PY@tok@o\endcsname{\def\PY@tc##1{\textcolor[rgb]{0.40,0.40,0.40}{##1}}}
\expandafter\def\csname PY@tok@ow\endcsname{\let\PY@bf=\textbf\def\PY@tc##1{\textcolor[rgb]{0.67,0.13,1.00}{##1}}}
\expandafter\def\csname PY@tok@nb\endcsname{\def\PY@tc##1{\textcolor[rgb]{0.00,0.50,0.00}{##1}}}
\expandafter\def\csname PY@tok@nf\endcsname{\def\PY@tc##1{\textcolor[rgb]{0.00,0.00,1.00}{##1}}}
\expandafter\def\csname PY@tok@nc\endcsname{\let\PY@bf=\textbf\def\PY@tc##1{\textcolor[rgb]{0.00,0.00,1.00}{##1}}}
\expandafter\def\csname PY@tok@nn\endcsname{\let\PY@bf=\textbf\def\PY@tc##1{\textcolor[rgb]{0.00,0.00,1.00}{##1}}}
\expandafter\def\csname PY@tok@ne\endcsname{\let\PY@bf=\textbf\def\PY@tc##1{\textcolor[rgb]{0.82,0.25,0.23}{##1}}}
\expandafter\def\csname PY@tok@nv\endcsname{\def\PY@tc##1{\textcolor[rgb]{0.10,0.09,0.49}{##1}}}
\expandafter\def\csname PY@tok@no\endcsname{\def\PY@tc##1{\textcolor[rgb]{0.53,0.00,0.00}{##1}}}
\expandafter\def\csname PY@tok@nl\endcsname{\def\PY@tc##1{\textcolor[rgb]{0.63,0.63,0.00}{##1}}}
\expandafter\def\csname PY@tok@ni\endcsname{\let\PY@bf=\textbf\def\PY@tc##1{\textcolor[rgb]{0.60,0.60,0.60}{##1}}}
\expandafter\def\csname PY@tok@na\endcsname{\def\PY@tc##1{\textcolor[rgb]{0.49,0.56,0.16}{##1}}}
\expandafter\def\csname PY@tok@nt\endcsname{\let\PY@bf=\textbf\def\PY@tc##1{\textcolor[rgb]{0.00,0.50,0.00}{##1}}}
\expandafter\def\csname PY@tok@nd\endcsname{\def\PY@tc##1{\textcolor[rgb]{0.67,0.13,1.00}{##1}}}
\expandafter\def\csname PY@tok@s\endcsname{\def\PY@tc##1{\textcolor[rgb]{0.73,0.13,0.13}{##1}}}
\expandafter\def\csname PY@tok@sd\endcsname{\let\PY@it=\textit\def\PY@tc##1{\textcolor[rgb]{0.73,0.13,0.13}{##1}}}
\expandafter\def\csname PY@tok@si\endcsname{\let\PY@bf=\textbf\def\PY@tc##1{\textcolor[rgb]{0.73,0.40,0.53}{##1}}}
\expandafter\def\csname PY@tok@se\endcsname{\let\PY@bf=\textbf\def\PY@tc##1{\textcolor[rgb]{0.73,0.40,0.13}{##1}}}
\expandafter\def\csname PY@tok@sr\endcsname{\def\PY@tc##1{\textcolor[rgb]{0.73,0.40,0.53}{##1}}}
\expandafter\def\csname PY@tok@ss\endcsname{\def\PY@tc##1{\textcolor[rgb]{0.10,0.09,0.49}{##1}}}
\expandafter\def\csname PY@tok@sx\endcsname{\def\PY@tc##1{\textcolor[rgb]{0.00,0.50,0.00}{##1}}}
\expandafter\def\csname PY@tok@m\endcsname{\def\PY@tc##1{\textcolor[rgb]{0.40,0.40,0.40}{##1}}}
\expandafter\def\csname PY@tok@gh\endcsname{\let\PY@bf=\textbf\def\PY@tc##1{\textcolor[rgb]{0.00,0.00,0.50}{##1}}}
\expandafter\def\csname PY@tok@gu\endcsname{\let\PY@bf=\textbf\def\PY@tc##1{\textcolor[rgb]{0.50,0.00,0.50}{##1}}}
\expandafter\def\csname PY@tok@gd\endcsname{\def\PY@tc##1{\textcolor[rgb]{0.63,0.00,0.00}{##1}}}
\expandafter\def\csname PY@tok@gi\endcsname{\def\PY@tc##1{\textcolor[rgb]{0.00,0.63,0.00}{##1}}}
\expandafter\def\csname PY@tok@gr\endcsname{\def\PY@tc##1{\textcolor[rgb]{1.00,0.00,0.00}{##1}}}
\expandafter\def\csname PY@tok@ge\endcsname{\let\PY@it=\textit}
\expandafter\def\csname PY@tok@gs\endcsname{\let\PY@bf=\textbf}
\expandafter\def\csname PY@tok@gp\endcsname{\let\PY@bf=\textbf\def\PY@tc##1{\textcolor[rgb]{0.00,0.00,0.50}{##1}}}
\expandafter\def\csname PY@tok@go\endcsname{\def\PY@tc##1{\textcolor[rgb]{0.53,0.53,0.53}{##1}}}
\expandafter\def\csname PY@tok@gt\endcsname{\def\PY@tc##1{\textcolor[rgb]{0.00,0.27,0.87}{##1}}}
\expandafter\def\csname PY@tok@err\endcsname{\def\PY@bc##1{\setlength{\fboxsep}{0pt}\fcolorbox[rgb]{1.00,0.00,0.00}{1,1,1}{\strut ##1}}}
\expandafter\def\csname PY@tok@kc\endcsname{\let\PY@bf=\textbf\def\PY@tc##1{\textcolor[rgb]{0.00,0.50,0.00}{##1}}}
\expandafter\def\csname PY@tok@kd\endcsname{\let\PY@bf=\textbf\def\PY@tc##1{\textcolor[rgb]{0.00,0.50,0.00}{##1}}}
\expandafter\def\csname PY@tok@kn\endcsname{\let\PY@bf=\textbf\def\PY@tc##1{\textcolor[rgb]{0.00,0.50,0.00}{##1}}}
\expandafter\def\csname PY@tok@kr\endcsname{\let\PY@bf=\textbf\def\PY@tc##1{\textcolor[rgb]{0.00,0.50,0.00}{##1}}}
\expandafter\def\csname PY@tok@bp\endcsname{\def\PY@tc##1{\textcolor[rgb]{0.00,0.50,0.00}{##1}}}
\expandafter\def\csname PY@tok@fm\endcsname{\def\PY@tc##1{\textcolor[rgb]{0.00,0.00,1.00}{##1}}}
\expandafter\def\csname PY@tok@vc\endcsname{\def\PY@tc##1{\textcolor[rgb]{0.10,0.09,0.49}{##1}}}
\expandafter\def\csname PY@tok@vg\endcsname{\def\PY@tc##1{\textcolor[rgb]{0.10,0.09,0.49}{##1}}}
\expandafter\def\csname PY@tok@vi\endcsname{\def\PY@tc##1{\textcolor[rgb]{0.10,0.09,0.49}{##1}}}
\expandafter\def\csname PY@tok@vm\endcsname{\def\PY@tc##1{\textcolor[rgb]{0.10,0.09,0.49}{##1}}}
\expandafter\def\csname PY@tok@sa\endcsname{\def\PY@tc##1{\textcolor[rgb]{0.73,0.13,0.13}{##1}}}
\expandafter\def\csname PY@tok@sb\endcsname{\def\PY@tc##1{\textcolor[rgb]{0.73,0.13,0.13}{##1}}}
\expandafter\def\csname PY@tok@sc\endcsname{\def\PY@tc##1{\textcolor[rgb]{0.73,0.13,0.13}{##1}}}
\expandafter\def\csname PY@tok@dl\endcsname{\def\PY@tc##1{\textcolor[rgb]{0.73,0.13,0.13}{##1}}}
\expandafter\def\csname PY@tok@s2\endcsname{\def\PY@tc##1{\textcolor[rgb]{0.73,0.13,0.13}{##1}}}
\expandafter\def\csname PY@tok@sh\endcsname{\def\PY@tc##1{\textcolor[rgb]{0.73,0.13,0.13}{##1}}}
\expandafter\def\csname PY@tok@s1\endcsname{\def\PY@tc##1{\textcolor[rgb]{0.73,0.13,0.13}{##1}}}
\expandafter\def\csname PY@tok@mb\endcsname{\def\PY@tc##1{\textcolor[rgb]{0.40,0.40,0.40}{##1}}}
\expandafter\def\csname PY@tok@mf\endcsname{\def\PY@tc##1{\textcolor[rgb]{0.40,0.40,0.40}{##1}}}
\expandafter\def\csname PY@tok@mh\endcsname{\def\PY@tc##1{\textcolor[rgb]{0.40,0.40,0.40}{##1}}}
\expandafter\def\csname PY@tok@mi\endcsname{\def\PY@tc##1{\textcolor[rgb]{0.40,0.40,0.40}{##1}}}
\expandafter\def\csname PY@tok@il\endcsname{\def\PY@tc##1{\textcolor[rgb]{0.40,0.40,0.40}{##1}}}
\expandafter\def\csname PY@tok@mo\endcsname{\def\PY@tc##1{\textcolor[rgb]{0.40,0.40,0.40}{##1}}}
\expandafter\def\csname PY@tok@ch\endcsname{\let\PY@it=\textit\def\PY@tc##1{\textcolor[rgb]{0.25,0.50,0.50}{##1}}}
\expandafter\def\csname PY@tok@cm\endcsname{\let\PY@it=\textit\def\PY@tc##1{\textcolor[rgb]{0.25,0.50,0.50}{##1}}}
\expandafter\def\csname PY@tok@cpf\endcsname{\let\PY@it=\textit\def\PY@tc##1{\textcolor[rgb]{0.25,0.50,0.50}{##1}}}
\expandafter\def\csname PY@tok@c1\endcsname{\let\PY@it=\textit\def\PY@tc##1{\textcolor[rgb]{0.25,0.50,0.50}{##1}}}
\expandafter\def\csname PY@tok@cs\endcsname{\let\PY@it=\textit\def\PY@tc##1{\textcolor[rgb]{0.25,0.50,0.50}{##1}}}

\def\PYZbs{\char`\\}
\def\PYZus{\char`\_}
\def\PYZob{\char`\{}
\def\PYZcb{\char`\}}
\def\PYZca{\char`\^}
\def\PYZam{\char`\&}
\def\PYZlt{\char`\<}
\def\PYZgt{\char`\>}
\def\PYZsh{\char`\#}
\def\PYZpc{\char`\%}
\def\PYZdl{\char`\$}
\def\PYZhy{\char`\-}
\def\PYZsq{\char`\'}
\def\PYZdq{\char`\"}
\def\PYZti{\char`\~}
% for compatibility with earlier versions
\def\PYZat{@}
\def\PYZlb{[}
\def\PYZrb{]}
\makeatother


    % Exact colors from NB
    \definecolor{incolor}{rgb}{0.0, 0.0, 0.5}
    \definecolor{outcolor}{rgb}{0.545, 0.0, 0.0}



    
    % Prevent overflowing lines due to hard-to-break entities
    \sloppy 
    % Setup hyperref package
    \hypersetup{
      breaklinks=true,  % so long urls are correctly broken across lines
      colorlinks=true,
      urlcolor=urlcolor,
      linkcolor=linkcolor,
      citecolor=citecolor,
      }
    % Slightly bigger margins than the latex defaults
    
    \geometry{verbose,tmargin=1in,bmargin=1in,lmargin=1in,rmargin=1in}
    
    

    \begin{document}
    
    
    \maketitle
    
    

    
    \begin{figure}
\centering
\includegraphics{Pictures/Adele-Hello.jpg}
\caption{hello}
\end{figure}

    \section{Topics}\label{topics}

\begin{itemize}
\tightlist
\item
  Data Presentation
\item
  Visualization
\item
  Optimization
\item
  Machine Learning
\item
  Misc.
\end{itemize}

    \section{Data Visualization}\label{data-visualization}

\subsection{This is a Jupyter Notebook with RISE
functionality}\label{this-is-a-jupyter-notebook-with-rise-functionality}

\begin{figure}
\centering
\includegraphics{Pictures/jupyter.png}
\caption{jupyter}
\end{figure}

    \subsubsection{You might have noticed I'm running this in my browser.
And yes, that does mean I can run it off a server (internal or external)
and show off results to other
people.}\label{you-might-have-noticed-im-running-this-in-my-browser.-and-yes-that-does-mean-i-can-run-it-off-a-server-internal-or-external-and-show-off-results-to-other-people.}

\begin{figure}
\centering
\includegraphics{Pictures/browser.jpg}
\caption{browser}
\end{figure}

    \subsubsection{Jupyter is nice in that it allows for multiple
programming languages. It's name is a portmanteau of Julia, Python and
R. I use Python, but it normally looks like
this:}\label{jupyter-is-nice-in-that-it-allows-for-multiple-programming-languages.-its-name-is-a-portmanteau-of-julia-python-and-r.-i-use-python-but-it-normally-looks-like-this}

\begin{figure}
\centering
\includegraphics{Pictures/Spyder.jpg}
\caption{spyder}
\end{figure}

    Python has a lot of useful interactions with other programming languages
that could be Hatch relevant

\begin{itemize}
\tightlist
\item
  SQL is the most obvious one. I can pull results from queries directly
  into Pandas dataframes for analysis
\item
  It also supports JS, so I could in theory run the D3.js libraries
  directly out of this for really nice visuals
\item
  It is possible to generate HTML with Python as well
\item
  Python programs can be made into applications which can be run on
  servers (e.g. live reporting)
\end{itemize}

    Python has several cool visualization tools

\begin{itemize}
\tightlist
\item
  Matplotlib
\item
  Seaborn
\item
  Bokeh
\end{itemize}

    \begin{Verbatim}[commandchars=\\\{\}]
{\color{incolor}In [{\color{incolor}1}]:} \PY{c+c1}{\PYZsh{}Here\PYZsq{}s a fairly basic matplotlib histogram. This is live code.}
        \PY{c+c1}{\PYZsh{}This is just setup to get data in}
        \PY{k+kn}{import} \PY{n+nn}{numpy} \PY{k}{as} \PY{n+nn}{np}
        \PY{k+kn}{import} \PY{n+nn}{matplotlib}\PY{n+nn}{.}\PY{n+nn}{pyplot} \PY{k}{as} \PY{n+nn}{plt}
        \PY{k+kn}{import} \PY{n+nn}{pandas} \PY{k}{as} \PY{n+nn}{pd}
        \PY{n}{id\PYZus{}df} \PY{o}{=} \PY{n}{pd}\PY{o}{.}\PY{n}{read\PYZus{}csv}\PY{p}{(}\PY{l+s+s1}{\PYZsq{}}\PY{l+s+s1}{C:/Users/Perry/.spyder\PYZhy{}py3/indeed\PYZus{}hw\PYZus{}data.csv}\PY{l+s+s1}{\PYZsq{}}\PY{p}{,} \PY{n}{header}\PY{o}{=}\PY{l+m+mi}{0}\PY{p}{)}
        \PY{n}{cols} \PY{o}{=} \PY{n}{id\PYZus{}df}\PY{o}{.}\PY{n}{columns}\PY{o}{.}\PY{n}{values}\PY{o}{.}\PY{n}{astype}\PY{p}{(}\PY{n+nb}{str}\PY{p}{)}
        \PY{n}{colct} \PY{o}{=} \PY{n}{np}\PY{o}{.}\PY{n}{arange}\PY{p}{(}\PY{n+nb}{len}\PY{p}{(}\PY{n}{cols}\PY{p}{)}\PY{p}{)}
        \PY{n}{nancol} \PY{o}{=} \PY{p}{[}\PY{p}{]}
        \PY{c+c1}{\PYZsh{}missing revenue the same as zero    }
        \PY{n}{id\PYZus{}df}\PY{p}{[}\PY{l+s+s1}{\PYZsq{}}\PY{l+s+s1}{revenue}\PY{l+s+s1}{\PYZsq{}}\PY{p}{]} \PY{o}{=} \PY{n}{id\PYZus{}df}\PY{p}{[}\PY{l+s+s1}{\PYZsq{}}\PY{l+s+s1}{revenue}\PY{l+s+s1}{\PYZsq{}}\PY{p}{]}\PY{o}{.}\PY{n}{fillna}\PY{p}{(}\PY{l+m+mi}{0}\PY{p}{)}
        \PY{n}{id\PYZus{}age\PYZus{}ex} \PY{o}{=} \PY{n}{id\PYZus{}df}\PY{o}{.}\PY{n}{loc}\PY{p}{[}\PY{n}{id\PYZus{}df}\PY{p}{[}\PY{l+s+s1}{\PYZsq{}}\PY{l+s+s1}{age}\PY{l+s+s1}{\PYZsq{}}\PY{p}{]} \PY{o}{\PYZlt{}}\PY{o}{=} \PY{l+m+mi}{0}\PY{p}{]}
        \PY{n}{id\PYZus{}age\PYZus{}ex\PYZus{}sub} \PY{o}{=} \PY{n}{id\PYZus{}age\PYZus{}ex}\PY{o}{.}\PY{n}{loc}\PY{p}{[}\PY{n}{id\PYZus{}age\PYZus{}ex}\PY{p}{[}\PY{l+s+s1}{\PYZsq{}}\PY{l+s+s1}{age}\PY{l+s+s1}{\PYZsq{}}\PY{p}{]} \PY{o}{\PYZlt{}} \PY{l+m+mi}{0}\PY{p}{]}  
        \PY{c+c1}{\PYZsh{}limit to only zero age    }
        \PY{n}{id\PYZus{}age\PYZus{}ex} \PY{o}{=} \PY{n}{id\PYZus{}age\PYZus{}ex}\PY{o}{.}\PY{n}{loc}\PY{p}{[}\PY{n}{id\PYZus{}age\PYZus{}ex}\PY{p}{[}\PY{l+s+s1}{\PYZsq{}}\PY{l+s+s1}{age}\PY{l+s+s1}{\PYZsq{}}\PY{p}{]} \PY{o}{==} \PY{l+m+mi}{0}\PY{p}{]}
        \PY{n}{id\PYZus{}df} \PY{o}{=} \PY{n}{id\PYZus{}df}\PY{o}{.}\PY{n}{loc}\PY{p}{[}\PY{n}{id\PYZus{}df}\PY{p}{[}\PY{l+s+s1}{\PYZsq{}}\PY{l+s+s1}{age}\PY{l+s+s1}{\PYZsq{}}\PY{p}{]} \PY{o}{\PYZgt{}}\PY{o}{=} \PY{l+m+mi}{0}\PY{p}{]}
        \PY{n}{id\PYZus{}df} \PY{o}{=} \PY{n}{id\PYZus{}df}\PY{o}{.}\PY{n}{loc}\PY{p}{[}\PY{n}{id\PYZus{}df}\PY{p}{[}\PY{l+s+s1}{\PYZsq{}}\PY{l+s+s1}{revenue}\PY{l+s+s1}{\PYZsq{}}\PY{p}{]} \PY{o}{\PYZlt{}} \PY{l+m+mi}{3500000000}\PY{p}{]}
            
            
\end{Verbatim}


    \begin{Verbatim}[commandchars=\\\{\}]
{\color{incolor}In [{\color{incolor}2}]:} \PY{c+c1}{\PYZsh{}Now I can draw the figure }
        \PY{n}{num\PYZus{}bins} \PY{o}{=} \PY{l+m+mi}{100}
        \PY{n}{gdt} \PY{o}{=} \PY{n}{id\PYZus{}df}\PY{p}{[}\PY{l+s+s1}{\PYZsq{}}\PY{l+s+s1}{revenue}\PY{l+s+s1}{\PYZsq{}}\PY{p}{]}\PY{o}{.}\PY{n}{loc}\PY{p}{[}\PY{n}{id\PYZus{}df}\PY{p}{[}\PY{l+s+s1}{\PYZsq{}}\PY{l+s+s1}{revenue}\PY{l+s+s1}{\PYZsq{}}\PY{p}{]} \PY{o}{\PYZgt{}} \PY{l+m+mi}{0}\PY{p}{]}
        \PY{n}{gdt} \PY{o}{=} \PY{n}{gdt}\PY{p}{[}\PY{n}{gdt} \PY{o}{\PYZlt{}}\PY{o}{=} \PY{l+m+mi}{150000000}\PY{p}{]}
        \PY{n}{n}\PY{p}{,} \PY{n}{bins}\PY{p}{,} \PY{n}{patches} \PY{o}{=} \PY{n}{plt}\PY{o}{.}\PY{n}{hist}\PY{p}{(}\PY{n}{gdt}\PY{p}{,} \PY{n}{num\PYZus{}bins}\PY{p}{,} \PY{n}{density} \PY{o}{=} \PY{l+m+mi}{1}\PY{p}{,} \PY{n}{facecolor}\PY{o}{=}\PY{l+s+s1}{\PYZsq{}}\PY{l+s+s1}{blue}\PY{l+s+s1}{\PYZsq{}}\PY{p}{,} \PY{n}{alpha}\PY{o}{=}\PY{l+m+mf}{0.5}\PY{p}{)}
        \PY{n}{plt}\PY{o}{.}\PY{n}{ticklabel\PYZus{}format}\PY{p}{(}\PY{n}{style}\PY{o}{=}\PY{l+s+s1}{\PYZsq{}}\PY{l+s+s1}{plain}\PY{l+s+s1}{\PYZsq{}}\PY{p}{,} \PY{n}{axis}\PY{o}{=}\PY{l+s+s1}{\PYZsq{}}\PY{l+s+s1}{x}\PY{l+s+s1}{\PYZsq{}}\PY{p}{,} \PY{n}{scilimits}\PY{o}{=}\PY{p}{(}\PY{l+m+mi}{0}\PY{p}{,}\PY{l+m+mi}{0}\PY{p}{)}\PY{p}{)}
        \PY{n}{plt}\PY{o}{.}\PY{n}{title}\PY{p}{(}\PY{l+s+s2}{\PYZdq{}}\PY{l+s+s2}{Revenue}\PY{l+s+s2}{\PYZdq{}}\PY{p}{)}
        \PY{n}{plt}\PY{o}{.}\PY{n}{xlabel}\PY{p}{(}\PY{l+s+s2}{\PYZdq{}}\PY{l+s+s2}{Value}\PY{l+s+s2}{\PYZdq{}}\PY{p}{)}
        \PY{n}{plt}\PY{o}{.}\PY{n}{ylabel}\PY{p}{(}\PY{l+s+s2}{\PYZdq{}}\PY{l+s+s2}{Frequency}\PY{l+s+s2}{\PYZdq{}}\PY{p}{)}
        \PY{n}{fig} \PY{o}{=} \PY{n}{plt}\PY{o}{.}\PY{n}{gcf}\PY{p}{(}\PY{p}{)}
        \PY{n}{fig}\PY{o}{.}\PY{n}{set\PYZus{}size\PYZus{}inches}\PY{p}{(}\PY{l+m+mf}{18.5}\PY{p}{,} \PY{l+m+mf}{10.5}\PY{p}{)}
\end{Verbatim}


    \begin{center}
    \adjustimage{max size={0.9\linewidth}{0.9\paperheight}}{The Notebook, but not that one starring Ryan Gosling _files/The Notebook, but not that one starring Ryan Gosling _8_0.png}
    \end{center}
    { \hspace*{\fill} \\}
    
    \begin{Verbatim}[commandchars=\\\{\}]
{\color{incolor}In [{\color{incolor}3}]:} \PY{c+c1}{\PYZsh{}We can quickly make this look nicer with Seaborn styles}
        \PY{k+kn}{import} \PY{n+nn}{seaborn} \PY{k}{as} \PY{n+nn}{sns}
        
        \PY{n}{sns}\PY{o}{.}\PY{n}{set}\PY{p}{(}\PY{p}{)}
        \PY{n}{n}\PY{p}{,} \PY{n}{bins}\PY{p}{,} \PY{n}{patches} \PY{o}{=} \PY{n}{plt}\PY{o}{.}\PY{n}{hist}\PY{p}{(}\PY{n}{gdt}\PY{p}{,} \PY{n}{num\PYZus{}bins}\PY{p}{,} \PY{n}{density} \PY{o}{=} \PY{l+m+mi}{1}\PY{p}{,} \PY{n}{alpha}\PY{o}{=}\PY{l+m+mf}{0.5}\PY{p}{)}
        \PY{n}{plt}\PY{o}{.}\PY{n}{title}\PY{p}{(}\PY{l+s+s2}{\PYZdq{}}\PY{l+s+s2}{Revenue}\PY{l+s+s2}{\PYZdq{}}\PY{p}{)}
        \PY{n}{plt}\PY{o}{.}\PY{n}{xlabel}\PY{p}{(}\PY{l+s+s2}{\PYZdq{}}\PY{l+s+s2}{Value}\PY{l+s+s2}{\PYZdq{}}\PY{p}{)}
        \PY{n}{plt}\PY{o}{.}\PY{n}{ylabel}\PY{p}{(}\PY{l+s+s2}{\PYZdq{}}\PY{l+s+s2}{Frequency}\PY{l+s+s2}{\PYZdq{}}\PY{p}{)}
        \PY{n}{fig} \PY{o}{=} \PY{n}{plt}\PY{o}{.}\PY{n}{gcf}\PY{p}{(}\PY{p}{)}
        \PY{n}{fig}\PY{o}{.}\PY{n}{set\PYZus{}size\PYZus{}inches}\PY{p}{(}\PY{l+m+mf}{18.5}\PY{p}{,} \PY{l+m+mf}{10.5}\PY{p}{)}
\end{Verbatim}


    \begin{center}
    \adjustimage{max size={0.9\linewidth}{0.9\paperheight}}{The Notebook, but not that one starring Ryan Gosling _files/The Notebook, but not that one starring Ryan Gosling _9_0.png}
    \end{center}
    { \hspace*{\fill} \\}
    
    \begin{Verbatim}[commandchars=\\\{\}]
{\color{incolor}In [{\color{incolor}4}]:} \PY{c+c1}{\PYZsh{}It can be used to create some really nice looking visualizations}
        \PY{c+c1}{\PYZsh{}This is an example from their site (again this is live code)}
        \PY{n}{sns}\PY{o}{.}\PY{n}{set}\PY{p}{(}\PY{n}{style}\PY{o}{=}\PY{l+s+s2}{\PYZdq{}}\PY{l+s+s2}{white}\PY{l+s+s2}{\PYZdq{}}\PY{p}{,} \PY{n}{rc}\PY{o}{=}\PY{p}{\PYZob{}}\PY{l+s+s2}{\PYZdq{}}\PY{l+s+s2}{axes.facecolor}\PY{l+s+s2}{\PYZdq{}}\PY{p}{:} \PY{p}{(}\PY{l+m+mi}{0}\PY{p}{,} \PY{l+m+mi}{0}\PY{p}{,} \PY{l+m+mi}{0}\PY{p}{,} \PY{l+m+mi}{0}\PY{p}{)}\PY{p}{\PYZcb{}}\PY{p}{)}
        \PY{n}{rs} \PY{o}{=} \PY{n}{np}\PY{o}{.}\PY{n}{random}\PY{o}{.}\PY{n}{RandomState}\PY{p}{(}\PY{l+m+mi}{1979}\PY{p}{)}
        \PY{n}{x} \PY{o}{=} \PY{n}{rs}\PY{o}{.}\PY{n}{randn}\PY{p}{(}\PY{l+m+mi}{500}\PY{p}{)}
        \PY{n}{g} \PY{o}{=} \PY{n}{np}\PY{o}{.}\PY{n}{tile}\PY{p}{(}\PY{n+nb}{list}\PY{p}{(}\PY{l+s+s2}{\PYZdq{}}\PY{l+s+s2}{ABCDEFGHIJ}\PY{l+s+s2}{\PYZdq{}}\PY{p}{)}\PY{p}{,} \PY{l+m+mi}{50}\PY{p}{)}
        \PY{n}{df} \PY{o}{=} \PY{n}{pd}\PY{o}{.}\PY{n}{DataFrame}\PY{p}{(}\PY{n+nb}{dict}\PY{p}{(}\PY{n}{x}\PY{o}{=}\PY{n}{x}\PY{p}{,} \PY{n}{g}\PY{o}{=}\PY{n}{g}\PY{p}{)}\PY{p}{)}
        \PY{n}{m} \PY{o}{=} \PY{n}{df}\PY{o}{.}\PY{n}{g}\PY{o}{.}\PY{n}{map}\PY{p}{(}\PY{n+nb}{ord}\PY{p}{)}
        \PY{n}{df}\PY{p}{[}\PY{l+s+s2}{\PYZdq{}}\PY{l+s+s2}{x}\PY{l+s+s2}{\PYZdq{}}\PY{p}{]} \PY{o}{+}\PY{o}{=} \PY{n}{m}
        \PY{n}{pal} \PY{o}{=} \PY{n}{sns}\PY{o}{.}\PY{n}{cubehelix\PYZus{}palette}\PY{p}{(}\PY{l+m+mi}{10}\PY{p}{,} \PY{n}{rot}\PY{o}{=}\PY{o}{\PYZhy{}}\PY{o}{.}\PY{l+m+mi}{25}\PY{p}{,} \PY{n}{light}\PY{o}{=}\PY{o}{.}\PY{l+m+mi}{7}\PY{p}{)}
        \PY{n}{g} \PY{o}{=} \PY{n}{sns}\PY{o}{.}\PY{n}{FacetGrid}\PY{p}{(}\PY{n}{df}\PY{p}{,} \PY{n}{row}\PY{o}{=}\PY{l+s+s2}{\PYZdq{}}\PY{l+s+s2}{g}\PY{l+s+s2}{\PYZdq{}}\PY{p}{,} \PY{n}{hue}\PY{o}{=}\PY{l+s+s2}{\PYZdq{}}\PY{l+s+s2}{g}\PY{l+s+s2}{\PYZdq{}}\PY{p}{,} \PY{n}{aspect}\PY{o}{=}\PY{l+m+mi}{15}\PY{p}{,} \PY{n}{size}\PY{o}{=}\PY{o}{.}\PY{l+m+mi}{5}\PY{p}{,} \PY{n}{palette}\PY{o}{=}\PY{n}{pal}\PY{p}{)}
        \PY{n}{g}\PY{o}{.}\PY{n}{map}\PY{p}{(}\PY{n}{sns}\PY{o}{.}\PY{n}{kdeplot}\PY{p}{,} \PY{l+s+s2}{\PYZdq{}}\PY{l+s+s2}{x}\PY{l+s+s2}{\PYZdq{}}\PY{p}{,} \PY{n}{clip\PYZus{}on}\PY{o}{=}\PY{k+kc}{False}\PY{p}{,} \PY{n}{shade}\PY{o}{=}\PY{k+kc}{True}\PY{p}{,} \PY{n}{alpha}\PY{o}{=}\PY{l+m+mi}{1}\PY{p}{,} \PY{n}{lw}\PY{o}{=}\PY{l+m+mf}{1.5}\PY{p}{,} \PY{n}{bw}\PY{o}{=}\PY{o}{.}\PY{l+m+mi}{2}\PY{p}{)}
        \PY{n}{g}\PY{o}{.}\PY{n}{map}\PY{p}{(}\PY{n}{sns}\PY{o}{.}\PY{n}{kdeplot}\PY{p}{,} \PY{l+s+s2}{\PYZdq{}}\PY{l+s+s2}{x}\PY{l+s+s2}{\PYZdq{}}\PY{p}{,} \PY{n}{clip\PYZus{}on}\PY{o}{=}\PY{k+kc}{False}\PY{p}{,} \PY{n}{color}\PY{o}{=}\PY{l+s+s2}{\PYZdq{}}\PY{l+s+s2}{w}\PY{l+s+s2}{\PYZdq{}}\PY{p}{,} \PY{n}{lw}\PY{o}{=}\PY{l+m+mi}{2}\PY{p}{,} \PY{n}{bw}\PY{o}{=}\PY{o}{.}\PY{l+m+mi}{2}\PY{p}{)}
        \PY{n}{g}\PY{o}{.}\PY{n}{map}\PY{p}{(}\PY{n}{plt}\PY{o}{.}\PY{n}{axhline}\PY{p}{,} \PY{n}{y}\PY{o}{=}\PY{l+m+mi}{0}\PY{p}{,} \PY{n}{lw}\PY{o}{=}\PY{l+m+mi}{2}\PY{p}{,} \PY{n}{clip\PYZus{}on}\PY{o}{=}\PY{k+kc}{False}\PY{p}{)}
        \PY{k}{def} \PY{n+nf}{label}\PY{p}{(}\PY{n}{x}\PY{p}{,} \PY{n}{color}\PY{p}{,} \PY{n}{label}\PY{p}{)}\PY{p}{:}
            \PY{n}{ax} \PY{o}{=} \PY{n}{plt}\PY{o}{.}\PY{n}{gca}\PY{p}{(}\PY{p}{)}
            \PY{n}{ax}\PY{o}{.}\PY{n}{text}\PY{p}{(}\PY{l+m+mi}{0}\PY{p}{,} \PY{o}{.}\PY{l+m+mi}{2}\PY{p}{,} \PY{n}{label}\PY{p}{,} \PY{n}{fontweight}\PY{o}{=}\PY{l+s+s2}{\PYZdq{}}\PY{l+s+s2}{bold}\PY{l+s+s2}{\PYZdq{}}\PY{p}{,} \PY{n}{color}\PY{o}{=}\PY{n}{color}\PY{p}{,} 
                    \PY{n}{ha}\PY{o}{=}\PY{l+s+s2}{\PYZdq{}}\PY{l+s+s2}{left}\PY{l+s+s2}{\PYZdq{}}\PY{p}{,} \PY{n}{va}\PY{o}{=}\PY{l+s+s2}{\PYZdq{}}\PY{l+s+s2}{center}\PY{l+s+s2}{\PYZdq{}}\PY{p}{,} \PY{n}{transform}\PY{o}{=}\PY{n}{ax}\PY{o}{.}\PY{n}{transAxes}\PY{p}{)}
        \PY{n}{g}\PY{o}{.}\PY{n}{map}\PY{p}{(}\PY{n}{label}\PY{p}{,} \PY{l+s+s2}{\PYZdq{}}\PY{l+s+s2}{x}\PY{l+s+s2}{\PYZdq{}}\PY{p}{)}
        \PY{n}{g}\PY{o}{.}\PY{n}{fig}\PY{o}{.}\PY{n}{subplots\PYZus{}adjust}\PY{p}{(}\PY{n}{hspace}\PY{o}{=}\PY{o}{\PYZhy{}}\PY{o}{.}\PY{l+m+mi}{25}\PY{p}{)}
        \PY{n}{g}\PY{o}{.}\PY{n}{set\PYZus{}titles}\PY{p}{(}\PY{l+s+s2}{\PYZdq{}}\PY{l+s+s2}{\PYZdq{}}\PY{p}{)}
        \PY{n}{g}\PY{o}{.}\PY{n}{set}\PY{p}{(}\PY{n}{yticks}\PY{o}{=}\PY{p}{[}\PY{p}{]}\PY{p}{)}
        \PY{n}{g}\PY{o}{.}\PY{n}{despine}\PY{p}{(}\PY{n}{bottom}\PY{o}{=}\PY{k+kc}{True}\PY{p}{,} \PY{n}{left}\PY{o}{=}\PY{k+kc}{True}\PY{p}{)}
\end{Verbatim}


\begin{Verbatim}[commandchars=\\\{\}]
{\color{outcolor}Out[{\color{outcolor}4}]:} <seaborn.axisgrid.FacetGrid at 0x28f13269668>
\end{Verbatim}
            
    \begin{center}
    \adjustimage{max size={0.9\linewidth}{0.9\paperheight}}{The Notebook, but not that one starring Ryan Gosling _files/The Notebook, but not that one starring Ryan Gosling _10_1.png}
    \end{center}
    { \hspace*{\fill} \\}
    
    \section{Bokeh is the one you'll be most interested in. It allows for
the same sort of interactivity that D3.js does, but is in Python. I'd
have to learn this, but it's a small lift since I do a lot of Python
anyway}\label{bokeh-is-the-one-youll-be-most-interested-in.-it-allows-for-the-same-sort-of-interactivity-that-d3.js-does-but-is-in-python.-id-have-to-learn-this-but-its-a-small-lift-since-i-do-a-lot-of-python-anyway}

    \begin{Verbatim}[commandchars=\\\{\}]
{\color{incolor}In [{\color{incolor}5}]:} \PY{c+c1}{\PYZsh{}Here\PYZsq{}s an example from their site}
        
        \PY{k+kn}{from} \PY{n+nn}{bokeh}\PY{n+nn}{.}\PY{n+nn}{io} \PY{k}{import} \PY{n}{output\PYZus{}notebook}
        \PY{k+kn}{from} \PY{n+nn}{bokeh}\PY{n+nn}{.}\PY{n+nn}{plotting} \PY{k}{import} \PY{n}{figure}\PY{p}{,} \PY{n}{show}
        
        \PY{n}{N} \PY{o}{=} \PY{l+m+mi}{4000}
        
        \PY{n}{x} \PY{o}{=} \PY{n}{np}\PY{o}{.}\PY{n}{random}\PY{o}{.}\PY{n}{random}\PY{p}{(}\PY{n}{size}\PY{o}{=}\PY{n}{N}\PY{p}{)} \PY{o}{*} \PY{l+m+mi}{100}
        \PY{n}{y} \PY{o}{=} \PY{n}{np}\PY{o}{.}\PY{n}{random}\PY{o}{.}\PY{n}{random}\PY{p}{(}\PY{n}{size}\PY{o}{=}\PY{n}{N}\PY{p}{)} \PY{o}{*} \PY{l+m+mi}{100}
        \PY{n}{radii} \PY{o}{=} \PY{n}{np}\PY{o}{.}\PY{n}{random}\PY{o}{.}\PY{n}{random}\PY{p}{(}\PY{n}{size}\PY{o}{=}\PY{n}{N}\PY{p}{)} \PY{o}{*} \PY{l+m+mf}{1.5}
        \PY{n}{colors} \PY{o}{=} \PY{p}{[}\PY{l+s+s2}{\PYZdq{}}\PY{l+s+s2}{\PYZsh{}}\PY{l+s+si}{\PYZpc{}02x}\PY{l+s+si}{\PYZpc{}02x}\PY{l+s+si}{\PYZpc{}02x}\PY{l+s+s2}{\PYZdq{}} \PY{o}{\PYZpc{}}\PY{p}{(}\PY{n+nb}{int}\PY{p}{(}\PY{n}{r}\PY{p}{)}\PY{p}{,}\PY{n+nb}{int}\PY{p}{(}\PY{n}{g}\PY{p}{)}\PY{p}{,}\PY{l+m+mi}{150}\PY{p}{)} \PY{k}{for} \PY{n}{r}\PY{p}{,} \PY{n}{g} \PY{o+ow}{in} \PY{n+nb}{zip}\PY{p}{(}\PY{n}{np}\PY{o}{.}\PY{n}{floor}\PY{p}{(}\PY{l+m+mi}{50}\PY{o}{+}\PY{l+m+mi}{2}\PY{o}{*}\PY{n}{x}\PY{p}{)}\PY{p}{,} \PY{n}{np}\PY{o}{.}\PY{n}{floor}\PY{p}{(}\PY{l+m+mi}{30}\PY{o}{+}\PY{l+m+mi}{2}\PY{o}{*}\PY{n}{y}\PY{p}{)}\PY{p}{)}\PY{p}{]}
        
        \PY{n}{output\PYZus{}notebook}\PY{p}{(}\PY{p}{)}
        
        \PY{n}{p} \PY{o}{=} \PY{n}{figure}\PY{p}{(}\PY{p}{)}
        \PY{n}{p}\PY{o}{.}\PY{n}{circle}\PY{p}{(}\PY{n}{x}\PY{p}{,} \PY{n}{y}\PY{p}{,} \PY{n}{radius} \PY{o}{=} \PY{n}{radii}\PY{p}{,} \PY{n}{fill\PYZus{}color}\PY{o}{=}\PY{n}{colors}\PY{p}{,} \PY{n}{fill\PYZus{}alpha}\PY{o}{=}\PY{l+m+mf}{0.6}\PY{p}{,} \PY{n}{line\PYZus{}color}\PY{o}{=}\PY{k+kc}{None}\PY{p}{)}
        
        \PY{n}{show}\PY{p}{(}\PY{n}{p}\PY{p}{)}
\end{Verbatim}


    
    
    
    
    
    
    
    
    \section{Optimization}\label{optimization}

\subsubsection{One of the more important skills I've picked up, Python
allows for calculating the type of advanced curve optimizations I used
to do in school with Matlab, but with access to better data structures
and file
IO}\label{one-of-the-more-important-skills-ive-picked-up-python-allows-for-calculating-the-type-of-advanced-curve-optimizations-i-used-to-do-in-school-with-matlab-but-with-access-to-better-data-structures-and-file-io}

\begin{itemize}
\tightlist
\item
  purchase quantities
\item
  bundles of items

  \begin{itemize}
  \tightlist
  \item
    creating packages that maximize our profit
  \item
    getting the most dense carts for customers
  \end{itemize}
\item
  labor allocations

  \begin{itemize}
  \tightlist
  \item
    CICs
  \item
    buildroom
  \end{itemize}
\item
  account distribution
\end{itemize}

    Smash.GG Fantasy

Given team budget and allowed to pick certain number of players. Very
similar to auction leagues for fantasy football

This is my team for the Norcal Regionals Event. I came in second:

\begin{figure}
\centering
\includegraphics{Pictures/fantasy.jpg}
\caption{NCR}
\end{figure}

    \section{I pick my teams by using linear optimization on player expected
values}\label{i-pick-my-teams-by-using-linear-optimization-on-player-expected-values}

\section{I composite expected values from prior results with some
manipulation based on my anecdotal knowledge of
players}\label{i-composite-expected-values-from-prior-results-with-some-manipulation-based-on-my-anecdotal-knowledge-of-players}

\section{There are multiple types of these optimizers beyond linear, but
linear works well for this problem space and compiles
quickly}\label{there-are-multiple-types-of-these-optimizers-beyond-linear-but-linear-works-well-for-this-problem-space-and-compiles-quickly}

    \begin{Verbatim}[commandchars=\\\{\}]
{\color{incolor}In [{\color{incolor}6}]:} \PY{c+c1}{\PYZsh{}This retroactively looked at the results from a major to calculate the best possible team}
        \PY{c+c1}{\PYZsh{}I use the same code with almost zero changes to evaluate teams; it\PYZsq{}s immaterial to switch}
        \PY{k+kn}{import} \PY{n+nn}{numpy} \PY{k}{as} \PY{n+nn}{np}
        \PY{k+kn}{import} \PY{n+nn}{pandas} \PY{k}{as} \PY{n+nn}{pd}
        \PY{k+kn}{from} \PY{n+nn}{pulp} \PY{k}{import} \PY{o}{*}
        
        \PY{n}{df} \PY{o}{=} \PY{n}{pd}\PY{o}{.}\PY{n}{read\PYZus{}csv}\PY{p}{(}\PY{l+s+s1}{\PYZsq{}}\PY{l+s+s1}{players1.csv}\PY{l+s+s1}{\PYZsq{}}\PY{p}{,} \PY{n}{header}\PY{o}{=}\PY{k+kc}{None}\PY{p}{)}
        \PY{n}{a} \PY{o}{=} \PY{n}{df}\PY{p}{[}\PY{l+m+mi}{0}\PY{p}{]}
        \PY{n}{b} \PY{o}{=} \PY{n}{df}\PY{p}{[}\PY{l+m+mi}{4}\PY{p}{]}
        \PY{n}{c} \PY{o}{=} \PY{n}{df}\PY{p}{[}\PY{l+m+mi}{5}\PY{p}{]}
        \PY{n}{tbl} \PY{o}{=} \PY{n}{np}\PY{o}{.}\PY{n}{array}\PY{p}{(}\PY{p}{[}\PY{n}{a}\PY{p}{,} \PY{n}{b}\PY{p}{,} \PY{n}{c}\PY{p}{]}\PY{p}{)}\PY{o}{.}\PY{n}{T}
        
        \PY{n}{fantasy\PYZus{}budget} \PY{o}{=} \PY{n}{pulp}\PY{o}{.}\PY{n}{LpProblem}\PY{p}{(}\PY{l+s+s1}{\PYZsq{}}\PY{l+s+s1}{fantasy optimization}\PY{l+s+s1}{\PYZsq{}}\PY{p}{,} \PY{n}{pulp}\PY{o}{.}\PY{n}{LpMaximize}\PY{p}{)}
        
        \PY{n}{players} \PY{o}{=} \PY{n}{tbl}\PY{p}{[}\PY{p}{:}\PY{p}{,}\PY{l+m+mi}{0}\PY{p}{]}
        
        \PY{n}{x} \PY{o}{=} \PY{n}{pulp}\PY{o}{.}\PY{n}{LpVariable}\PY{o}{.}\PY{n}{dict}\PY{p}{(}\PY{l+s+s1}{\PYZsq{}}\PY{l+s+s1}{x\PYZus{}}\PY{l+s+si}{\PYZpc{}s}\PY{l+s+s1}{\PYZsq{}}\PY{p}{,} \PY{n}{players}\PY{p}{,} \PY{n}{lowBound} \PY{o}{=}\PY{l+m+mi}{0}\PY{p}{,} \PY{n}{cat}\PY{o}{=}\PY{l+s+s1}{\PYZsq{}}\PY{l+s+s1}{Integer}\PY{l+s+s1}{\PYZsq{}}\PY{p}{)}
        
        \PY{n}{price} \PY{o}{=} \PY{n+nb}{dict}\PY{p}{(}\PY{n+nb}{zip}\PY{p}{(}\PY{n}{players}\PY{p}{,} \PY{n}{tbl}\PY{p}{[}\PY{p}{:}\PY{p}{,}\PY{l+m+mi}{2}\PY{p}{]}\PY{p}{)}\PY{p}{)}
        
        \PY{n}{ev} \PY{o}{=} \PY{n+nb}{dict}\PY{p}{(}\PY{n+nb}{zip}\PY{p}{(}\PY{n}{players}\PY{p}{,} \PY{n}{tbl}\PY{p}{[}\PY{p}{:}\PY{p}{,}\PY{l+m+mi}{1}\PY{p}{]}\PY{p}{)}\PY{p}{)}
        
        \PY{n}{fantasy\PYZus{}budget} \PY{o}{+}\PY{o}{=} \PY{n+nb}{sum}\PY{p}{(}\PY{p}{[} \PY{p}{(}\PY{n}{x}\PY{p}{[}\PY{n}{i}\PY{p}{]}\PY{o}{*}\PY{n}{price}\PY{p}{[}\PY{n}{i}\PY{p}{]}\PY{p}{)}\PY{o}{*}\PY{n}{ev}\PY{p}{[}\PY{n}{i}\PY{p}{]} \PY{k}{for} \PY{n}{i} \PY{o+ow}{in} \PY{n}{players}\PY{p}{]}\PY{p}{)}
        
        \PY{n}{fantasy\PYZus{}budget} \PY{o}{+}\PY{o}{=} \PY{n+nb}{sum}\PY{p}{(}\PY{p}{[} \PY{p}{(}\PY{n}{x}\PY{p}{[}\PY{n}{i}\PY{p}{]}\PY{o}{*}\PY{n}{price}\PY{p}{[}\PY{n}{i}\PY{p}{]}\PY{p}{)} \PY{k}{for} \PY{n}{i} \PY{o+ow}{in} \PY{n}{players}\PY{p}{]}\PY{p}{)} \PY{o}{\PYZlt{}}\PY{o}{=} \PY{l+m+mi}{1200}
        
        \PY{n}{fantasy\PYZus{}budget} \PY{o}{+}\PY{o}{=} \PY{n+nb}{sum}\PY{p}{(}\PY{p}{[} \PY{p}{(}\PY{n}{x}\PY{p}{[}\PY{n}{i}\PY{p}{]}\PY{p}{)} \PY{k}{for} \PY{n}{i} \PY{o+ow}{in} \PY{n}{players}\PY{p}{]}\PY{p}{)} \PY{o}{==} \PY{l+m+mi}{12}
           
        \PY{k}{for} \PY{n}{player} \PY{o+ow}{in} \PY{n}{players}\PY{p}{:}
            \PY{n}{fantasy\PYZus{}budget} \PY{o}{+}\PY{o}{=} \PY{n}{x}\PY{p}{[}\PY{n}{player}\PY{p}{]} \PY{o}{\PYZlt{}}\PY{o}{=} \PY{n}{price}\PY{p}{[}\PY{n}{player}\PY{p}{]}    
          
        \PY{k}{for} \PY{n}{player} \PY{o+ow}{in} \PY{n}{players}\PY{p}{:}
            \PY{n}{fantasy\PYZus{}budget} \PY{o}{+}\PY{o}{=} \PY{n}{x}\PY{p}{[}\PY{n}{player}\PY{p}{]} \PY{o}{\PYZlt{}}\PY{o}{=} \PY{l+m+mi}{1}       
        
        \PY{n}{fantasy\PYZus{}budget}\PY{o}{.}\PY{n}{solve}\PY{p}{(}\PY{p}{)}
        
        \PY{n}{hld} \PY{o}{=} \PY{p}{[}\PY{p}{]}
        \PY{k}{for} \PY{n}{player} \PY{o+ow}{in} \PY{n}{players}\PY{p}{:}
            \PY{n}{t} \PY{o}{=} \PY{n}{x}\PY{p}{[}\PY{n}{player}\PY{p}{]}
            \PY{n}{u} \PY{o}{=} \PY{n}{x}\PY{p}{[}\PY{n}{player}\PY{p}{]}\PY{o}{.}\PY{n}{value}\PY{p}{(}\PY{p}{)}
            \PY{n}{v} \PY{o}{=} \PY{n}{np}\PY{o}{.}\PY{n}{array}\PY{p}{(}\PY{p}{[}\PY{n}{t}\PY{p}{,} \PY{n}{u}\PY{p}{]}\PY{p}{)}
            \PY{n}{hld}\PY{o}{.}\PY{n}{append}\PY{p}{(}\PY{n}{v}\PY{p}{)}
        
        \PY{n}{df} \PY{o}{=} \PY{n}{pd}\PY{o}{.}\PY{n}{DataFrame}\PY{p}{(}\PY{n}{hld}\PY{p}{,} \PY{n}{index}\PY{o}{=}\PY{k+kc}{None}\PY{p}{,} \PY{n}{columns}\PY{o}{=}\PY{k+kc}{None}\PY{p}{)}  
        
        \PY{n+nb}{print}\PY{p}{(}\PY{n}{df}\PY{p}{[}\PY{p}{(}\PY{n}{df}\PY{p}{[}\PY{l+m+mi}{1}\PY{p}{]} \PY{o}{\PYZgt{}} \PY{l+m+mi}{0}\PY{p}{)}\PY{p}{]}\PY{p}{)}
\end{Verbatim}


    \begin{Verbatim}[commandchars=\\\{\}]
                   0    1
0         x\_SonicFox  1.0
1              x\_GO1  1.0
2           x\_Dogura  1.0
3         x\_NyChrisG  1.0
4         x\_Kazunoko  1.0
30        x\_TwiTchAy  1.0
44  x\_PSYKENonTWITCH  1.0
46            x\_Des!  1.0
47            x\_Nice  1.0
61      x\_Coolestred  1.0
62   x\_Coffee\_prince  1.0
63          x\_Shogun  1.0

    \end{Verbatim}

    \section{This team would have performed \textgreater{}25\% better than
the single best human picked team even without bonus
questions}\label{this-team-would-have-performed-25-better-than-the-single-best-human-picked-team-even-without-bonus-questions}

\section{I of course got too clever for my own good and forced a player
on my team and did not finish in prizes for this
event}\label{i-of-course-got-too-clever-for-my-own-good-and-forced-a-player-on-my-team-and-did-not-finish-in-prizes-for-this-event}

    \begin{Verbatim}[commandchars=\\\{\}]
{\color{incolor}In [{\color{incolor}7}]:} \PY{k+kn}{from} \PY{n+nn}{IPython}\PY{n+nn}{.}\PY{n+nn}{display} \PY{k}{import} \PY{n}{HTML}
        
        \PY{c+c1}{\PYZsh{} Youtube}
        \PY{n}{HTML}\PY{p}{(}\PY{l+s+s1}{\PYZsq{}}\PY{l+s+s1}{\PYZlt{}iframe width=}\PY{l+s+s1}{\PYZdq{}}\PY{l+s+s1}{560}\PY{l+s+s1}{\PYZdq{}}\PY{l+s+s1}{ height=}\PY{l+s+s1}{\PYZdq{}}\PY{l+s+s1}{315}\PY{l+s+s1}{\PYZdq{}}\PY{l+s+s1}{ src=}\PY{l+s+s1}{\PYZdq{}}\PY{l+s+s1}{https://www.youtube.com/embed/vP65VVoUm6E}\PY{l+s+s1}{\PYZdq{}}\PY{l+s+s1}{ frameborder=}\PY{l+s+s1}{\PYZdq{}}\PY{l+s+s1}{0}\PY{l+s+s1}{\PYZdq{}}\PY{l+s+s1}{ allow=}\PY{l+s+s1}{\PYZdq{}}\PY{l+s+s1}{autoplay; encrypted\PYZhy{}media}\PY{l+s+s1}{\PYZdq{}}\PY{l+s+s1}{ allowfullscreen\PYZgt{}\PYZlt{}/iframe\PYZgt{}}\PY{l+s+s1}{\PYZsq{}}\PY{p}{)}
\end{Verbatim}


\begin{Verbatim}[commandchars=\\\{\}]
{\color{outcolor}Out[{\color{outcolor}7}]:} <IPython.core.display.HTML object>
\end{Verbatim}
            
    \section{Machine Learning}\label{machine-learning}

    \section{Recently a friend invited me to do some Optical Character
Recognition work to help process match VODs into a
repository}\label{recently-a-friend-invited-me-to-do-some-optical-character-recognition-work-to-help-process-match-vods-into-a-repository}

\subsection{The best players of the fighting game Guilty Gear live in
Japan and play at the Mikado
arcade}\label{the-best-players-of-the-fighting-game-guilty-gear-live-in-japan-and-play-at-the-mikado-arcade}

\begin{figure}
\centering
\includegraphics{Pictures/mikado.jpg}
\caption{mikado}
\end{figure}

\subsection{In the West we watch these VODs to learn how to play better.
Unfortunately, what we get are unmarked videos of several hours in
length that lack critical
information}\label{in-the-west-we-watch-these-vods-to-learn-how-to-play-better.-unfortunately-what-we-get-are-unmarked-videos-of-several-hours-in-length-that-lack-critical-information}

    \section{Relevant Information}\label{relevant-information}

\begin{itemize}
\tightlist
\item
  Characters
\item
  Match Start Time
\item
  Players
\end{itemize}

\subsubsection{Luckily for me work has been done on all 3, but the
current solution to the third is a paid web service (Google Vision
API)}\label{luckily-for-me-work-has-been-done-on-all-3-but-the-current-solution-to-the-third-is-a-paid-web-service-google-vision-api}

\subsubsection{Also fortunately, all revelant information is available
on one screen, even though quality is quite
low}\label{also-fortunately-all-revelant-information-is-available-on-one-screen-even-though-quality-is-quite-low}

\begin{figure}
\centering
\includegraphics{Pictures/screen3.jpg}
\caption{screen}
\end{figure}

    \section{I was going to evaluate the relative performance between a
local solution called PyTesseract to Google Vision API (which is world
class
performance)}\label{i-was-going-to-evaluate-the-relative-performance-between-a-local-solution-called-pytesseract-to-google-vision-api-which-is-world-class-performance}

\subsubsection{Tesseract is pretty good, and the new version uses LSTM
trained sets, but it's pretty far behind Google theoretically as google
is applying very sophisticated pre-processing to a cluster trained model
based on 8+ layer convolutional neural networks
(CNN)}\label{tesseract-is-pretty-good-and-the-new-version-uses-lstm-trained-sets-but-its-pretty-far-behind-google-theoretically-as-google-is-applying-very-sophisticated-pre-processing-to-a-cluster-trained-model-based-on-8-layer-convolutional-neural-networks-cnn}

\subsubsection{Process Image}\label{process-image}

\begin{figure}
\centering
\includegraphics{Pictures/3656.png}
\caption{proc}
\end{figure}

    \begin{Verbatim}[commandchars=\\\{\}]
{\color{incolor}In [{\color{incolor}8}]:} \PY{k+kn}{from} \PY{n+nn}{PIL} \PY{k}{import} \PY{n}{Image}
        \PY{k+kn}{import} \PY{n+nn}{pytesseract}
        \PY{k+kn}{import} \PY{n+nn}{argparse}
        \PY{k+kn}{import} \PY{n+nn}{cv2}
        \PY{k+kn}{import} \PY{n+nn}{os}
        
        \PY{n}{pytesseract}\PY{o}{.}\PY{n}{pytesseract}\PY{o}{.}\PY{n}{tesseract\PYZus{}cmd} \PY{o}{=} \PY{l+s+s1}{\PYZsq{}}\PY{l+s+s1}{C:/Program Files (x86)/Tesseract\PYZhy{}OCR/tesseract.exe}\PY{l+s+s1}{\PYZsq{}}
        
        \PY{n}{ap} \PY{o}{=} \PY{n}{argparse}\PY{o}{.}\PY{n}{ArgumentParser}\PY{p}{(}\PY{p}{)}
        \PY{n}{ap}\PY{o}{.}\PY{n}{add\PYZus{}argument}\PY{p}{(}\PY{l+s+s2}{\PYZdq{}}\PY{l+s+s2}{\PYZhy{}i}\PY{l+s+s2}{\PYZdq{}}\PY{p}{,} \PY{l+s+s2}{\PYZdq{}}\PY{l+s+s2}{\PYZhy{}\PYZhy{}image}\PY{l+s+s2}{\PYZdq{}}\PY{p}{,} \PY{n}{required}\PY{o}{=}\PY{k+kc}{True}\PY{p}{,}
        	\PY{n}{help}\PY{o}{=}\PY{l+s+s2}{\PYZdq{}}\PY{l+s+s2}{path to input image to be OCR}\PY{l+s+s2}{\PYZsq{}}\PY{l+s+s2}{d}\PY{l+s+s2}{\PYZdq{}}\PY{p}{)}
        \PY{n}{ap}\PY{o}{.}\PY{n}{add\PYZus{}argument}\PY{p}{(}\PY{l+s+s2}{\PYZdq{}}\PY{l+s+s2}{\PYZhy{}p}\PY{l+s+s2}{\PYZdq{}}\PY{p}{,} \PY{l+s+s2}{\PYZdq{}}\PY{l+s+s2}{\PYZhy{}\PYZhy{}preprocess}\PY{l+s+s2}{\PYZdq{}}\PY{p}{,} \PY{n+nb}{type}\PY{o}{=}\PY{n+nb}{str}\PY{p}{,} \PY{n}{default}\PY{o}{=}\PY{k+kc}{None}\PY{p}{,}
        	\PY{n}{help}\PY{o}{=}\PY{l+s+s2}{\PYZdq{}}\PY{l+s+s2}{type of preprocessing to be done}\PY{l+s+s2}{\PYZdq{}}\PY{p}{)}
        \PY{n}{args} \PY{o}{=} \PY{n+nb}{vars}\PY{p}{(}\PY{n}{ap}\PY{o}{.}\PY{n}{parse\PYZus{}args}\PY{p}{(}\PY{p}{)}\PY{p}{)}
        
        \PY{c+c1}{\PYZsh{}filename = os.path.join(skimage.data\PYZus{}dir, \PYZsq{}moon.png\PYZsq{})}
        
        \PY{c+c1}{\PYZsh{}filename = \PYZsq{}C:/Users/Perry/Pictures/screen3.jpg\PYZsq{}}
        \PY{n}{image} \PY{o}{=} \PY{n}{cv2}\PY{o}{.}\PY{n}{imread}\PY{p}{(}\PY{n}{args}\PY{p}{[}\PY{l+s+s2}{\PYZdq{}}\PY{l+s+s2}{image}\PY{l+s+s2}{\PYZdq{}}\PY{p}{]}\PY{p}{)}
        \PY{n}{image} \PY{o}{=} \PY{n}{cv2}\PY{o}{.}\PY{n}{resize}\PY{p}{(}\PY{n}{image}\PY{p}{,} \PY{p}{(}\PY{l+m+mi}{1280}\PY{p}{,}\PY{l+m+mi}{720}\PY{p}{)}\PY{p}{,} \PY{n}{interpolation} \PY{o}{=} \PY{n}{cv2}\PY{o}{.}\PY{n}{INTER\PYZus{}AREA}\PY{p}{)}
        \PY{n}{image2} \PY{o}{=} \PY{n}{image}\PY{p}{[}\PY{l+m+mi}{492}\PY{p}{:}\PY{l+m+mi}{524}\PY{p}{,} \PY{l+m+mi}{46}\PY{p}{:}\PY{l+m+mi}{409}\PY{p}{]}
        \PY{n}{image} \PY{o}{=} \PY{n}{image}\PY{p}{[}\PY{l+m+mi}{480}\PY{p}{:}\PY{l+m+mi}{514}\PY{p}{,} \PY{l+m+mi}{825}\PY{p}{:}\PY{l+m+mi}{1118}\PY{p}{]}
        \PY{n}{imagec} \PY{o}{=} \PY{p}{[}\PY{p}{]}
        \PY{n}{imagec} \PY{o}{=} \PY{n}{imagec}\PY{o}{.}\PY{n}{append}\PY{p}{(}\PY{n}{image}\PY{p}{)}
        \PY{c+c1}{\PYZsh{}imagec = imagec.append(image2)}
        \PY{n}{gray} \PY{o}{=} \PY{n}{cv2}\PY{o}{.}\PY{n}{cvtColor}\PY{p}{(}\PY{n}{image}\PY{p}{,} \PY{n}{cv2}\PY{o}{.}\PY{n}{COLOR\PYZus{}BGR2GRAY}\PY{p}{)}
        
        
        \PY{k}{if} \PY{n}{args}\PY{p}{[}\PY{l+s+s2}{\PYZdq{}}\PY{l+s+s2}{preprocess}\PY{l+s+s2}{\PYZdq{}}\PY{p}{]} \PY{o}{==} \PY{l+s+s2}{\PYZdq{}}\PY{l+s+s2}{thresh}\PY{l+s+s2}{\PYZdq{}}\PY{p}{:}
        	\PY{n}{gray} \PY{o}{=} \PY{n}{cv2}\PY{o}{.}\PY{n}{threshold}\PY{p}{(}\PY{n}{gray}\PY{p}{,} \PY{l+m+mi}{140}\PY{p}{,} \PY{l+m+mi}{255}\PY{p}{,}
        		\PY{n}{cv2}\PY{o}{.}\PY{n}{THRESH\PYZus{}TOZERO}\PY{p}{)}\PY{p}{[}\PY{l+m+mi}{1}\PY{p}{]} \PY{c+c1}{\PYZsh{}| cv2.THRESH\PYZus{}OTSU)[1]}
        
        \PY{c+c1}{\PYZsh{} noise}
        \PY{k}{elif} \PY{n}{args}\PY{p}{[}\PY{l+s+s2}{\PYZdq{}}\PY{l+s+s2}{preprocess}\PY{l+s+s2}{\PYZdq{}}\PY{p}{]} \PY{o}{==} \PY{l+s+s2}{\PYZdq{}}\PY{l+s+s2}{blur}\PY{l+s+s2}{\PYZdq{}}\PY{p}{:}
        	\PY{n}{gray} \PY{o}{=} \PY{n}{cv2}\PY{o}{.}\PY{n}{medianBlur}\PY{p}{(}\PY{n}{gray}\PY{p}{,} \PY{l+m+mi}{3}\PY{p}{)}
         
        
        \PY{n}{gray} \PY{o}{=} \PY{n}{util}\PY{o}{.}\PY{n}{invert}\PY{p}{(}\PY{n}{gray}\PY{p}{)}
        
        \PY{c+c1}{\PYZsh{}gray = equalize\PYZus{}hist(gray)}
        
        \PY{n}{filename} \PY{o}{=} \PY{l+s+s2}{\PYZdq{}}\PY{l+s+si}{\PYZob{}\PYZcb{}}\PY{l+s+s2}{.png}\PY{l+s+s2}{\PYZdq{}}\PY{o}{.}\PY{n}{format}\PY{p}{(}\PY{n}{os}\PY{o}{.}\PY{n}{getpid}\PY{p}{(}\PY{p}{)}\PY{p}{)}
        \PY{n}{cv2}\PY{o}{.}\PY{n}{imwrite}\PY{p}{(}\PY{n}{filename}\PY{p}{,} \PY{n}{gray}\PY{p}{)}
        
        
        \PY{n}{text} \PY{o}{=} \PY{n}{pytesseract}\PY{o}{.}\PY{n}{image\PYZus{}to\PYZus{}string}\PY{p}{(}\PY{n}{Image}\PY{o}{.}\PY{n}{open}\PY{p}{(}\PY{n}{filename}\PY{p}{)}\PY{p}{,} \PY{n}{lang}\PY{o}{=}\PY{l+s+s1}{\PYZsq{}}\PY{l+s+s1}{jpn+eng}\PY{l+s+s1}{\PYZsq{}}\PY{p}{)}\PY{c+c1}{\PYZsh{}, config=\PYZsq{}\PYZhy{}\PYZhy{}psm 10\PYZsq{})}
        \PY{n}{os}\PY{o}{.}\PY{n}{remove}\PY{p}{(}\PY{n}{filename}\PY{p}{)}
        \PY{n+nb}{print}\PY{p}{(}\PY{n}{text}\PY{p}{)}
         
        \PY{c+c1}{\PYZsh{} show the output images}
        \PY{n}{cv2}\PY{o}{.}\PY{n}{imshow}\PY{p}{(}\PY{l+s+s2}{\PYZdq{}}\PY{l+s+s2}{Image}\PY{l+s+s2}{\PYZdq{}}\PY{p}{,} \PY{n}{image}\PY{p}{)}
        \PY{n}{cv2}\PY{o}{.}\PY{n}{imshow}\PY{p}{(}\PY{l+s+s2}{\PYZdq{}}\PY{l+s+s2}{Output}\PY{l+s+s2}{\PYZdq{}}\PY{p}{,} \PY{n}{gray}\PY{p}{)}
        \PY{n}{cv2}\PY{o}{.}\PY{n}{waitKey}\PY{p}{(}\PY{l+m+mi}{0}\PY{p}{)}
\end{Verbatim}


    \begin{Verbatim}[commandchars=\\\{\}]
usage: ipykernel\_launcher.py [-h] -i IMAGE [-p PREPROCESS]
ipykernel\_launcher.py: error: the following arguments are required: -i/--image

    \end{Verbatim}

    \begin{Verbatim}[commandchars=\\\{\}]

        An exception has occurred, use \%tb to see the full traceback.
    

        SystemExit: 2
    

    \end{Verbatim}

    \begin{Verbatim}[commandchars=\\\{\}]
C:\textbackslash{}Users\textbackslash{}Perry\textbackslash{}AppData\textbackslash{}Local\textbackslash{}conda\textbackslash{}conda\textbackslash{}envs\textbackslash{}Classification\textbackslash{}lib\textbackslash{}site-packages\textbackslash{}IPython\textbackslash{}core\textbackslash{}interactiveshell.py:2918: UserWarning: To exit: use 'exit', 'quit', or Ctrl-D.
  warn("To exit: use 'exit', 'quit', or Ctrl-D.", stacklevel=1)

    \end{Verbatim}

    \section{This is a command line application, so I can't run it in this
notebook, but this is basically what it
outputs}\label{this-is-a-command-line-application-so-i-cant-run-it-in-this-notebook-but-this-is-basically-what-it-outputs}

\begin{figure}
\centering
\includegraphics{Pictures/days.jpg}
\caption{days}
\end{figure}

\begin{itemize}
\tightlist
\item
  Text output for use in creating upload script
\item
  Unprocessed mask
\item
  Processed mask
\end{itemize}

    \subsubsection{How does this help
Hatch?}\label{how-does-this-help-hatch}

\subsubsection{Hatch has a paper process in order entry, or often has
things written down that need to go into the CRM/ERP in text form that
currently have to be typed
in}\label{hatch-has-a-paper-process-in-order-entry-or-often-has-things-written-down-that-need-to-go-into-the-crmerp-in-text-form-that-currently-have-to-be-typed-in}

\subsubsection{Current commercial solutions are usually poor performers
or require API knowledge. I can cover the latter and enable use of
Google and MS' world class systems (they are actually quite reasonable,
a year of scanning forms for Hatch would probably cost less than
10,000USD)}\label{current-commercial-solutions-are-usually-poor-performers-or-require-api-knowledge.-i-can-cover-the-latter-and-enable-use-of-google-and-ms-world-class-systems-they-are-actually-quite-reasonable-a-year-of-scanning-forms-for-hatch-would-probably-cost-less-than-10000usd}

\subsubsection{It doesnt even matter if it's typed or handwritten; I
have ML strategies for handling that because I can train datasets based
on Hatch employee's handwriting which would
significantly}\label{it-doesnt-even-matter-if-its-typed-or-handwritten-i-have-ml-strategies-for-handling-that-because-i-can-train-datasets-based-on-hatch-employees-handwriting-which-would-significantly}

\subsubsection{I can't match Google's precision, but I can beat
Tesseract by using my own CNNs. In fact, I've trained my own using one
of
these:}\label{i-cant-match-googles-precision-but-i-can-beat-tesseract-by-using-my-own-cnns.-in-fact-ive-trained-my-own-using-one-of-these}

\begin{figure}
\centering
\includegraphics{Pictures/1080ti.jpg}
\caption{teneighty}
\end{figure}

    \section{ML more generally}\label{ml-more-generally}

    \subsection{So I've spent the last year on
ML}\label{so-ive-spent-the-last-year-on-ml}

\subsection{Specifically, I've done signal classification of EEG signals
for the purposes of detecting seizures in real
time}\label{specifically-ive-done-signal-classification-of-eeg-signals-for-the-purposes-of-detecting-seizures-in-real-time}

\subsection{Considering how SSG works, ML may be the only way to
approximate the correct order of SSG
games.}\label{considering-how-ssg-works-ml-may-be-the-only-way-to-approximate-the-correct-order-of-ssg-games.}

\subsection{Because I built SSG progress into a SQL table (i.e. a
2-dimensional matrix), I can iteratively use each column to estimate the
score kids would get on another
game}\label{because-i-built-ssg-progress-into-a-sql-table-i.e.-a-2-dimensional-matrix-i-can-iteratively-use-each-column-to-estimate-the-score-kids-would-get-on-another-game}

\subsection{Basically it's ripping off Netflix's recommendation system.
It really doesn't matter what specific engine you use either as it's a
pretty low-complexity problem, so you can use XGBoost or another tabular
engine that runs quickly on CPUs to avoid additional hardware
cost}\label{basically-its-ripping-off-netflixs-recommendation-system.-it-really-doesnt-matter-what-specific-engine-you-use-either-as-its-a-pretty-low-complexity-problem-so-you-can-use-xgboost-or-another-tabular-engine-that-runs-quickly-on-cpus-to-avoid-additional-hardware-cost}

    \subsection{For what it's worth, the same thing can be applied to
recommending items to customers based on the length of time since
they've purchased. It would be stronger if Hatch had more proprietary
stuff, but it should probably work out
anyway}\label{for-what-its-worth-the-same-thing-can-be-applied-to-recommending-items-to-customers-based-on-the-length-of-time-since-theyve-purchased.-it-would-be-stronger-if-hatch-had-more-proprietary-stuff-but-it-should-probably-work-out-anyway}

\subsection{TheDialer/Skynet could be improved with ML by updating
estimators based on call
success}\label{thedialerskynet-could-be-improved-with-ml-by-updating-estimators-based-on-call-success}

\subsection{There could theoretically be a Dragon
v3}\label{there-could-theoretically-be-a-dragon-v3}

\subsection{Since my work has mainly been in classification, I would
want to get ML based predictions of account Expected
Values}\label{since-my-work-has-mainly-been-in-classification-i-would-want-to-get-ml-based-predictions-of-account-expected-values}

    \section{Miscellaneous}\label{miscellaneous}

\subsubsection{Image Classification}\label{image-classification}

\subsubsection{Signal Processing
specifically}\label{signal-processing-specifically}

\subsubsection{Obviously I still have the skills to do reporting and
tackle the same sort of SQL problems I did before (e.g. renewals revenue
reporting)}\label{obviously-i-still-have-the-skills-to-do-reporting-and-tackle-the-same-sort-of-sql-problems-i-did-before-e.g.-renewals-revenue-reporting}

\subsubsection{Questions?}\label{questions}


    % Add a bibliography block to the postdoc
    
    
    
    \end{document}
